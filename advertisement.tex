\documentclass[a4paper,12pt]{article}
%\usepackage{stdpage}
\usepackage{microtype}
\usepackage{tikz}
\usetikzlibrary{calc}
\usetikzlibrary{through}
\usepackage{tgpagella}
\usepackage{subfig}
\usepackage[utf8]{inputenc}
\usepackage[unicode,breaklinks=true,hypertexnames=false]{hyperref}
\usepackage{eurosym}
\DeclareUnicodeCharacter{20AC}{\euro}

\definecolor{mylinkcolor}{rgb}{0,0,0.5}
\hypersetup{%
   colorlinks=true,%
   linkcolor=mylinkcolor,%
   urlcolor=mylinkcolor,%
}

\begin{document}

\vspace*{3cm}

\begin{center}
{
    \huge Games for a multi-touch table

    \vspace{0.5em}

    \large IT project – Advertisement
}

\vspace{3em}

Petr Viktorin

University of Eastern Finland

2011
\end{center}

\vspace*{2.718\fill}

\newpage

Introducing an open-source collection of games for a multi-touch table.

\hspace{1em}

The project contains two two-player games designed to take advantage of
true multitouch devices.

The first game, Maze, has one player navigate a maze, while his or her opponent
is rebuilding the maze to make it more difficult.
The players take turns playing the two roles.

The second game, Towers, is a tower-defense-style game. Each player controls
half of the screen and tries to hold off an infestation of his or her base by
strategically placing towers that shoot at invading alien creatures.

The games are designed for simplicity and engaging gameplay.
It is easy to learn how to play, making them suitable for casual gamers.
Requiring players to interact on a common touchscreen, brings elements of
traditional board games, such as natural communication between players, to the
interactivity and fast action of computer gaming.

The project also contains a system for logging the games, making it possible to
replay and analyze them later.
This makes the games suitable for human-computer interaction research involving
multi-touch technology.

The games can be run on a low-cost multitouch table, built for about €100 from
parts.
Instructions for building such a table are available as part of the report.
Other true multitouch devices can also be used.
The project is designed to run on Linux, Windows and Mac OS X, although
installation on Linux is easiest.

The software is written in the Python language, using various related
technologies.
It is released under a liberal open-source licence, inviting contributions from
interested programmers or artists.

\end{document}
